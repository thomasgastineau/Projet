\documentclass[a4paper,11pt]{report}
\usepackage[T1]{fontenc} % pour écrire en français
\usepackage[francais]{babel} %pour écrire en français
\usepackage[utf8x]{inputenc} %encodage en UTF-8
\usepackage{fancyhdr} %pour gérer les en-têtes et pieds de page
\usepackage{amsmath,amscd,amssymb} %pour insérer des expressions scientifiques
\usepackage[pdftex]{graphicx} %pour inclure des figures
\usepackage{subfig}
\usepackage{hyperref} %pour créer des liens hyper-textes
\usepackage{verbatim} %pour citer du code Latex ou autre
\usepackage{url} %pour citer une adresse web
\pagenumbering{arabic} %type de numérotation des pages
\graphicspath{{Figures/}} %les figures sont rangées dans le dossier Figures
\pagestyle{plain} %style des pages

%%%%%%%%%%%%%%%%%%%%%%%%%%%%%%%%%%%%%%%%%%%%%%%%%%%%%%%%%%%%%%%%%%%%

%----------------------------------------------------------------------------------------------------------
%				PAGE DE GARDE
%----------------------------------------------------------------------------------------------------------

\title{Projet Lévitation}
\date{Année universitaire :  2014-2015}
\author{Simon Hamery, Flavien Le Bailly, Thomas Gastineau}
 


%%%%%%%%%%%%%%%%%%%%%%%%%%%%%%%%%%%%%%%%%%%%%%%%%%%%%%%%%%%%%%%%%%%%%%%%%%%%%%%%%%%%%%%%%%%%%%%%%%%%%%%%%%%%%%%%
\begin{document}
\maketitle  %génère la page de garde
\newpage  %comme son nom l'indique ...

%%%%%%%%%%%%%%%%%%%%%%%%%%%%%%%%%%%%%%%%%%%%%%%%%%%%%%%%%
\pagenumbering{roman} \setcounter{page}{1} %les pages commencent à être numérotées en lettre romaines.
\chapter*{Remerciements} %le premier chapitre concerne les remerciements. L'étoile * signifie que le chapitre ne sera pas numéroté et n'apparaitra donc pas dans la table des matières
Nous remercions ...
%%%%%%%%%%%%%%%%%%%%%%%%%%%%%%%%%%%%%%%%%%%%%%%%%%%%%%%%%

\newpage
\null
\thispagestyle{empty}
\newpage

 %------------------------------------------------------------------------------------------------------
 %					    TABLE DES MATIÈRES 
 %-----------------------------------------------------------------------------------------------------
{\tableofcontents} 
%\newpage\
\listoffigures


\newpage

%%%%%%%%%%%%%%%%%%%%%%%%%%%%%%%%%%%%%%%%%%%%%%%%%%%%%%%%%%%%%%%%%%%%%%%%%%%%%%%%%%%%%%%%%%%%%%%%%%%%%%%%%%%%%%%%
\chapter*{Introduction}
\addcontentsline{toc}{chapter}{Introduction} %Ce chapitre n'est pas numéroté mais apparaitra dans la table des matières gràce à cette commande.
\pagenumbering{arabic} \setcounter{page}{1} %Le numéro de page est remis à zéro et la numérotation est en chiffre arabe


 


%%%%%%%%%%%%%%%%%%%%%%%%%%%%%%%%%%%%%%%%%%%%%%%%%%%%%%%%%%%%%%%%%%%%%%%%%%%%%%%%%%%%%%%%%%%%%%%%%%%%%%%%%%%%%%%%
\chapter{Principe}
%%%%%%%%%%%%%%%%%%%%%%%%%%%%%%%%%%%%%%%%%%%%%%%%%%%%%%%%%%%%%%%%%%%%%%%%%%%%%%%%%%%%%%%%%%%%%%%%%%%%%%%%%%%%%%%%
\chapter{Expériences}

\section{Mise en place d'un champs d'ondes stationnaires}

\subsection{Protocole}

\subsection{Résultats}

\subsection{Conclusion}

\section{Essai de lévitation dans un tube}

\chapter{Conclusion}

\end{document}